%----------------------------------------------------------------------------------------
%	Packages and margins
%----------------------------------------------------------------------------------------
\documentclass[12pt]{article}
\usepackage{fancyhdr} % Required for custom headers
\usepackage{lastpage} % Required to determine the last page for the footer
\usepackage{extramarks} % Required for headers and footers
\usepackage[dvipsnames]{color} % Required for custom colors
\usepackage{graphicx} % Required to insert images
\usepackage{listings} % Required for insertion of code
\usepackage[margin=1in]{geometry} % Page setup
\usepackage[disable]{todonotes}
\usepackage{setspace,epstopdf,amsmath,amsfonts,amssymb,amsthm}
\usepackage{marginnote,datetime,enumitem,subfigure,rotating,fancyvrb}
\usepackage{hyperref,float}
\usepackage{booktabs}
\usepackage{multirow}
\usepackage{graphics}
\usdate
% Margins
\topmargin=-0.45in
\evensidemargin=0in
\oddsidemargin=0in
\textwidth=6.5in
\textheight=9.0in
\headsep=0.25in
%----------------------------------------------------------------------------------------
% Header and footer set up
%----------------------------------------------------------------------------------------
\pagestyle{fancy}
\lhead{\hmwkAuthorName} % Top left header
\chead{\hmwkClass: \hmwkTitle} % Top center head
\rhead{\firstxmark} % Top right header
\lfoot{\lastxmark} % Bottom left footer
\cfoot{} % Bottom center footer
\rfoot{Page\ \thepage\ of\ \protect\pageref{LastPage}} % Bottom right footer
\renewcommand\headrulewidth{0.4pt} % Size of the header rule
\renewcommand\footrulewidth{0.4pt} % Size of the footer rule
%----------------------------------------------------------------------------------------
% Useful commands
%----------------------------------------------------------------------------------------
% Number equations within align* environment
\newcommand\numberthis{\addtocounter{equation}{1}\tag{\theequation}}
% Bold vectors
\renewcommand{\vec}[1]{\mathbf{#1}}
% Indepedence symbol
\newcommand\independent{\protect\mathpalette{\protect\independenT}{\perp}}
\def\independenT#1#2{\mathrel{\rlap{$#1#2$}\mkern2mu{#1#2}}}
% How large to make matrices
\setcounter{MaxMatrixCols}{20}
% Removes indentation from paragraphs
\setlength\parindent{0pt} 
%----------------------------------------------------------------------------------------
% Page structure commands - please ignore
%----------------------------------------------------------------------------------------
% Header and footer when a page split occurs within a problem environment
\newcommand{\enterProblemHeader}[1]{
\nobreak\extramarks{#1}{#1 continued on next page\ldots}\nobreak
\nobreak\extramarks{#1 (continued)}{#1 continued on next page\ldots}\nobreak
}
% Header and footer for when a page split occurs between problem environments
\newcommand{\exitProblemHeader}[1]{
\nobreak\extramarks{#1 (continued)}{#1 continued on next page\ldots}\nobreak
\nobreak\extramarks{#1}{}\nobreak
}
\newcounter{homeworkProblemCounter} % Creates a counter to keep track of the number of problems
\newcommand{\homeworkProblemName}{}
\newenvironment{homeworkProblem}[1][Problem \arabic{homeworkProblemCounter}]{ % Makes a new environment called homeworkProblem which takes 1 argument (custom name) but the default is "Problem #"
\stepcounter{homeworkProblemCounter} % Increase counter for number of problems
\renewcommand{\homeworkProblemName}{#1} % Assign \homeworkProblemName the name of the problem
\enterProblemHeader{\homeworkProblemName} % Header and footer within the environment
}{
\exitProblemHeader{\homeworkProblemName} % Header and footer after the environment
}
\newcommand{\problemAnswer}[1]{ % Defines the problem answer command with the content as the only argument
\noindent\framebox[\columnwidth][c]{\begin{minipage}{0.98\columnwidth}#1\end{minipage}} % Makes the box around the problem answer and puts the content inside
}
\newcommand{\homeworkSectionName}{}
\newenvironment{homeworkSection}[1]{ % New environment for sections within homework problems, takes 1 argument - the name of the section
\renewcommand{\homeworkSectionName}{#1} % Assign \homeworkSectionName to the name of the section from the environment argument
\subsection{\homeworkSectionName} % Make a subsection with the custom name of the subsection
\enterProblemHeader{\homeworkProblemName\ [\homeworkSectionName]} % Header and footer within the environment
}{
\enterProblemHeader{\homeworkProblemName} % Header and footer after the environment
}
%----------------------------------------------------------------------------------------
%	Student name and class
%----------------------------------------------------------------------------------------
\newcommand{\hmwkTitle}{Problem Set\ \#5} % Assignment title
\newcommand{\hmwkClass}{BST\ 234} % Course/class
\newcommand{\hmwkAuthorName}{Student name} % Your name
%----------------------------------------------------------------------------------------
\begin{document}
%----------------------------------------------------------------------------------------
%	PROBLEM 1
%----------------------------------------------------------------------------------------
\begin{homeworkProblem}
\section{}
Which of the following two ways to compute a polynomial is stable? Which one is the more efficient algorithm?
\begin{enumerate}
\item $p(x) = a_0 + a_1x + \dots + a_nx^n, x \in \mathbb{R}$
\item $p(x) = a_0 + x(a_1 + x(a_2 + x(a_3 + \dots + x (a_{n-1} + x a_n) \dots )))$
\end{enumerate}
\textit{Solution:}\\

\end{homeworkProblem}
%----------------------------------------------------------------------------------------
% PROBLEM 2
%----------------------------------------------------------------------------------------
\begin{homeworkProblem}
\section{}
Compute the condition number for the following two matrices. \\ \\
\begin{minipage}{0.4\textwidth}
\[
A = 
\begin{bmatrix}
    2 & 1 & 1 \\
    1 & 2 & 1 \\
    1 & 1 & 2
\end{bmatrix}
\]
\end{minipage}%
\begin{minipage}{0.5\textwidth}
\[
B = 
\begin{bmatrix}
    168 & 113  \\
    113 & 76
\end{bmatrix}
\]
\end{minipage} \\

\textit{Solution:}\\ 

\end{homeworkProblem}
%----------------------------------------------------------------------------------------
% PROBLEM 3
%----------------------------------------------------------------------------------------


\begin{homeworkProblem}
\section{}
Please write matrix (1) (slide 3  in BST\_234\_Numerical\_Aspects\_of\_Algorithms March 4) in:
\begin{enumerate}
\item triplet format
\item csr format
\end{enumerate}
\end{homeworkProblem}

\begin{homeworkProblem}
\section{}
 Given 2 sparse ($n \times n$)-times matrices, A and B. What is the expected sparsity of $A + B $ and $A \times B$
 \end{homeworkProblem}
 
 \begin{homeworkProblem}
 \section{}
 For genotype matrices, define a new csr format that takes the special data structure of genotype data into account.
 \begin{itemize}
 \item Using simulation studies (by drawing from the provided MAF data), how much storage space do you save on average?
 \item At least how much storage space do you save in 95\% of the cases/simulation studies?
 \end{itemize}
  \end{homeworkProblem}

%----------------------------------------------------------------------------------------
\end{document}
